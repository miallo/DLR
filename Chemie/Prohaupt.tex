% Für Bindekorrektur als optionales Argument "BCORfaktormitmaßeinheit", dann
% sieht auch Option "twoside" vernünftig aus
% Näheres zu "scrartcl" bzw. "scrreprt" und "scrbook" siehe KOMA-Skript Doku
\documentclass[12pt,a4paper,titlepage,headinclude,bibtotoc]{scrartcl}


%---- Allgemeine Layout Einstellungen ------------------------------------------

% Für Kopf und Fußzeilen, siehe auch KOMA-Skript Doku
\usepackage[komastyle]{scrpage2}
\pagestyle{scrheadings}
\setheadsepline{0.5pt}[\color{black}]
\automark[section]{chapter}


%Einstellungen für Figuren- und Tabellenbeschriftungen
\setkomafont{captionlabel}{\sffamily\bfseries}
\setcapindent{0em}


%---- Weitere Pakete -----------------------------------------------------------
% Die Pakete sind alle in der TeX Live Distribution enthalten. Wichtige Adressen
% www.ctan.org, www.dante.de

% Sprachunterstützung
\usepackage[ngerman]{babel}

% Benutzung von Umlauten direkt im Text
% entweder "latin1" oder "utf8"
\usepackage[utf8]{inputenc}

% Pakete mit Mathesymbolen und zur Beseitigung von Schwächen der Mathe-Umgebung
\usepackage{latexsym,exscale,stmaryrd,amssymb,amsmath}

% Weitere Symbole
\usepackage[nointegrals]{wasysym}
\usepackage{eurosym}

% Anderes Literaturverzeichnisformat
%\usepackage[square,sort&compress]{natbib}

% Für Farbe
\usepackage{color}

% Zur Graphikausgabe
%Beipiel: \includegraphics[width=\textwidth]{grafik.png}
\usepackage{graphicx}

% Text umfließt Graphiken und Tabellen
% Beispiel:
% \begin{wrapfigure}[Zeilenanzahl]{"l" oder "r"}{breite}
%   \centering
%   \includegraphics[width=...]{grafik}
%   \caption{Beschriftung} 
%   \label{fig:grafik}
% \end{wrapfigure}
\usepackage{wrapfig}

% Mehrere Abbildungen nebeneinander
% Beispiel:
% \begin{figure}[htb]
%   \centering
%   \subfigure[Beschriftung 1\label{fig:label1}]
%   {\includegraphics[width=0.49\textwidth]{grafik1}}
%   \hfill
%   \subfigure[Beschriftung 2\label{fig:label2}]
%   {\includegraphics[width=0.49\textwidth]{grafik2}}
%   \caption{Beschriftung allgemein}
%   \label{fig:label-gesamt}
% \end{figure}
\usepackage{subfigure}

% Caption neben Abbildung
% Beispiel:
% \sidecaptionvpos{figure}{"c" oder "t" oder "b"}
% \begin{SCfigure}[rel. Breite (normalerweise = 1)][hbt]
%   \centering
%   \includegraphics[width=0.5\textwidth]{grafik.png}
%   \caption{Beschreibung}
%   \label{fig:}
% \end{SCfigure}
\usepackage{sidecap}

% Befehl für "Entspricht"-Zeichen
\newcommand{\corresponds}{\ensuremath{\mathrel{\widehat{=}}}}
% Befehl für Errorfunction
\newcommand{\erf}[1]{\text{ erf}\ensuremath{\left( #1 \right)}}

%Fußnoten zwingend auf diese Seite setzen
\interfootnotelinepenalty=1000

%Für chemische Formeln (von www.dante.de)
%% Anpassung an LaTeX(2e) von Bernd Raichle
\makeatletter
\DeclareRobustCommand{\chemical}[1]{%
  {\(\m@th
   \edef\resetfontdimens{\noexpand\)%
       \fontdimen16\textfont2=\the\fontdimen16\textfont2
       \fontdimen17\textfont2=\the\fontdimen17\textfont2\relax}%
   \fontdimen16\textfont2=2.7pt \fontdimen17\textfont2=2.7pt
   \mathrm{#1}%
   \resetfontdimens}}
\makeatother

%Honecker-Kasten mit $$\shadowbox{$xxxx$}$$
\usepackage{fancybox}

%SI-Package
\usepackage{siunitx}

%keine Einrückung, wenn Latex doppelte Leerzeile
\parindent0pt

%Bibliography \bibliography{literatur} und \cite{gerthsen}
%\usepackage{cite}
\usepackage{babelbib}
\selectbiblanguage{ngerman}

\begin{document}

\begin{titlepage}
\centering
\textsc{\Large Vermittlung strömungsphysikalischer Vorgänge im Experiment,
\\[1.5ex] Universität Göttingen}

\vspace*{3cm}

\rule{\textwidth}{1pt}\\[0.5cm]
{\huge \bfseries
  Versuch Chemie  \\[1.5ex]
  Protokoll}\\[0.5cm]
\rule{\textwidth}{1pt}

\vspace*{3cm}

\begin{Large}
\begin{tabular}{ll}
Praktikant: &  Michael Lohmann\\
% &  Felix Kurtz\\
% &  Kevin Lüdemann\\
% &  Skrollan Detzler\\
 E-Mail: & m.lohmann@stud.uni-goettingen.de\\
% &  felix.kurtz@stud.uni-goettingen.de\\
% &  kevin.luedemann@stud.uni-goettingen.de\\
 Versuchsdatum: & 18.1.2016\\
\end{tabular}
\end{Large}

\vspace*{0.8cm}

\begin{Large}
\fbox{
  \begin{minipage}[t][2.5cm][t]{6cm} 
    Testat:
  \end{minipage}
}
\end{Large}

\end{titlepage}

\tableofcontents

\newpage


\section{Einleitung}
Da nicht immer genau die Menge an benötigter elektrischer Energie zur verfügung steht, die benötigt wird und gerade regenerative Energien keine konstante Leistung bringen, sind Energiespeicher wichtig.
Da jegliche Umwandlung zwischen Energieformen immer mit Verlusten verbunden sind und da auch Stromspeicher nur einen geringen Wirkungsgrad haben, wäre eine Speicherung von Wärme wünschenswert.

\section{sensible Wärmespeicher}
Sensible Wärmespeicher sind die einfachste Form von Speichern.
Dabei wird eine Substanz (zum Beispiel Wasser oder ein Stein) erhitzt.
Die Wärme wird (bestimmt durch die Wärmespeicherkapazität) in die Erhitzung der Masse umgesetzt.
Da die Speicher zwar günstig sind, jedoch nach Außen hin viel Wärme abstrahlen, sind sie nicht geeignet.

Schaut man sich die Temperaturkurve an, wie sich der Körper bei wärmezufuhr erwärmt,so sieht man ein konstantes Wachstum.

\section{Latentwärmespeicher}
Latentwärmespeicher sind Speicher, bei denen ein Teil der Wärme in einen Phasenübergang umgesetzt wird.
Als Beispiel wäre es, Wasser zum kochen zu bringen und den Wasserdampf zu speichern.
Da bei einem Phasenübergang viel Energie benötigt wird, kann das Speichervolumen bei gleicher Kapazität wesentlich geringer sein.
Das wohl bekannteste Beispiel hierfür sind Handwärmekissen, welche im "`Normalzustand"' durchsichtig und flüssig sind und ganz normal gelagert werden können.
Aktiviert man sie jedoch durch das Knicken eines Plättchens, so erwärmen sie sich auf $58\si\celsius$ und werden fest.

Fügt man so einem Speicher Energie zu, so kann man feststellen, dass die Temperatur zunächst wie beim sensiblen gradlinig ansteigt, kommt es jedoch zur Phasenumwandlung, so bleibt die Temperatur solange konstant, bis alles in der anderen Phase ist.




\section{Thermochemischer Wärmespeicher}
Die Speicherung, die am wenigsten Energie verliert durch das Lagern ist die thermochemische.
Diese wandelt die Wärme in einer chemischen Reaktion in verschiedene Stoffe um.
Das bekannteste Beispiel hier sind Trockenpäckchen, welche sehr hydrophil sind.
Was vermutlich weniger bekannt ist, ist dass durch das Binden von Wasser Energie frei wird, da der Stoff mit Kristallwasser weniger potentielle Energie im Gitter besitzt.
Sind die Trockenpäckchen trocken, so sind die Kügelchen blau und werden weiß/durchsichtig (wie man sie kennt), wenn sie Feuchtigkeit aufgenommen haben.

Interessant ist, dass der Farbverlauf von trocken zu feucht bei <+***************Kupfersulfat******************+> genau anders herum ist.
Dieser Stoff ist vielen vielleicht schon aus der Schule bekannt, wo man eine farblose Substanz bekommt, welche sich bei einigen Tropfen Wasser blau färbt.
Hält man das Reagenzglas jedoch wieder über den Bunsenbrenner, so entfärbt sich der Stoff wieder.
Beobachtet man den kälteren oberen Rand des Reagenzglases, so sieht man, wie sich Wasser niederschlägt.
Dies war zuvor im Kristall als Kristallwasser eingebunden.





\end{document}
