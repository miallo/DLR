% Für Bindekorrektur als optionales Argument "BCORfaktormitmaßeinheit", dann
% sieht auch Option "twoside" vernünftig aus
% Näheres zu "scrartcl" bzw. "scrreprt" und "scrbook" siehe KOMA-Skript Doku
\documentclass[12pt,a4paper,titlepage,headinclude,bibtotoc]{scrartcl}


%---- Allgemeine Layout Einstellungen ------------------------------------------

% Für Kopf und Fußzeilen, siehe auch KOMA-Skript Doku
\usepackage[komastyle]{scrpage2}
\pagestyle{scrheadings}
\setheadsepline{0.5pt}[\color{black}]
\automark[section]{chapter}


%Einstellungen für Figuren- und Tabellenbeschriftungen
\setkomafont{captionlabel}{\sffamily\bfseries}
\setcapindent{0em}


%---- Weitere Pakete -----------------------------------------------------------
% Die Pakete sind alle in der TeX Live Distribution enthalten. Wichtige Adressen
% www.ctan.org, www.dante.de

% Sprachunterstützung
\usepackage[ngerman]{babel}

% Benutzung von Umlauten direkt im Text
% entweder "latin1" oder "utf8"
\usepackage[utf8]{inputenc}

% Pakete mit Mathesymbolen und zur Beseitigung von Schwächen der Mathe-Umgebung
\usepackage{latexsym,exscale,stmaryrd,amssymb,amsmath}

% Weitere Symbole
\usepackage[nointegrals]{wasysym}
\usepackage{eurosym}

% Anderes Literaturverzeichnisformat
%\usepackage[square,sort&compress]{natbib}

% Für Farbe
\usepackage{color}

% Zur Graphikausgabe
%Beipiel: \includegraphics[width=\textwidth]{grafik.png}
\usepackage{graphicx}

% Text umfließt Graphiken und Tabellen
% Beispiel:
% \begin{wrapfigure}[Zeilenanzahl]{"l" oder "r"}{breite}
%   \centering
%   \includegraphics[width=...]{grafik}
%   \caption{Beschriftung} 
%   \label{fig:grafik}
% \end{wrapfigure}
\usepackage{wrapfig}

% Mehrere Abbildungen nebeneinander
% Beispiel:
% \begin{figure}[htb]
%   \centering
%   \subfigure[Beschriftung 1\label{fig:label1}]
%   {\includegraphics[width=0.49\textwidth]{grafik1}}
%   \hfill
%   \subfigure[Beschriftung 2\label{fig:label2}]
%   {\includegraphics[width=0.49\textwidth]{grafik2}}
%   \caption{Beschriftung allgemein}
%   \label{fig:label-gesamt}
% \end{figure}
\usepackage{subfigure}

% Caption neben Abbildung
% Beispiel:
% \sidecaptionvpos{figure}{"c" oder "t" oder "b"}
% \begin{SCfigure}[rel. Breite (normalerweise = 1)][hbt]
%   \centering
%   \includegraphics[width=0.5\textwidth]{grafik.png}
%   \caption{Beschreibung}
%   \label{fig:}
% \end{SCfigure}
\usepackage{sidecap}

% Befehl für "Entspricht"-Zeichen
\newcommand{\corresponds}{\ensuremath{\mathrel{\widehat{=}}}}
% Befehl für Errorfunction
\newcommand{\erf}[1]{\text{ erf}\ensuremath{\left( #1 \right)}}

%Fußnoten zwingend auf diese Seite setzen
\interfootnotelinepenalty=1000

%Für chemische Formeln (von www.dante.de)
%% Anpassung an LaTeX(2e) von Bernd Raichle
\makeatletter
\DeclareRobustCommand{\chemical}[1]{%
  {\(\m@th
   \edef\resetfontdimens{\noexpand\)%
       \fontdimen16\textfont2=\the\fontdimen16\textfont2
       \fontdimen17\textfont2=\the\fontdimen17\textfont2\relax}%
   \fontdimen16\textfont2=2.7pt \fontdimen17\textfont2=2.7pt
   \mathrm{#1}%
   \resetfontdimens}}
\makeatother

%Honecker-Kasten mit $$\shadowbox{$xxxx$}$$
\usepackage{fancybox}

%SI-Package
\usepackage{siunitx}

%keine Einrückung, wenn Latex doppelte Leerzeile
\parindent0pt

%Bibliography \bibliography{literatur} und \cite{gerthsen}
%\usepackage{cite}
\usepackage{babelbib}
\selectbiblanguage{ngerman}

\begin{document}

\begin{titlepage}
\centering
\textsc{\Large Vermittlung strömungsphysikalischer Vorgänge im Experiment,
\\[1.5ex] Universität Göttingen}

\vspace*{3cm}

\rule{\textwidth}{1pt}\\[0.5cm]
{\huge \bfseries
  Versuch Lärm  \\[1.5ex]
  Protokoll}\\[0.5cm]
\rule{\textwidth}{1pt}

\vspace*{3cm}

\begin{Large}
\begin{tabular}{ll}
Praktikant: &  Michael Lohmann\\
% &  Felix Kurtz\\
% &  Kevin Lüdemann\\
% &  Skrollan Detzler\\
 E-Mail: & m.lohmann@stud.uni-goettingen.de\\
% &  felix.kurtz@stud.uni-goettingen.de\\
% &  kevin.luedemann@stud.uni-goettingen.de\\
 Betreuer: & \\
 Versuchsdatum: & 07.12.2015\\
\end{tabular}
\end{Large}

\vspace*{0.8cm}

\begin{Large}
\fbox{
  \begin{minipage}[t][2.5cm][t]{6cm} 
    Testat:
  \end{minipage}
}
\end{Large}

\end{titlepage}

\tableofcontents

\newpage

\section{Einleitung}
\label{sec:einleitung}
Lärm spielt in der Umgebung der Menschen, gerade was die Lebensqualität anbelangt, eine entscheidene Rolle.
Desshalb ist es sinnvoll, verschieden Methoden zur Aufspürung von Schallquellen kennenzulernen.

\section{Akustische Kamera}
Eine Schallkamera ist eine Kamera, welche gekoppelt ist mit eimen Array aus Mikrophonen.
Diese sind möglichst unregelmäßig angeordnet, damit möglichst viele Informationen aus den Daten gewonnen werden können.
Sonst könnte bei einer bestimmten Wellenlänge eventuell kaum eine Aussage getroffen werden, wo der Schall herkommt.
Funktionieren tut die Kamera, indem sie aus geringen Laufzeitunterschieden und Phasenverschiebungen berechnet, aus welcher Richtung der Schall gekommen sein muss.
Eine Videoaufnahme wird nun mit dieser Information überlagert und die lauten Bereiche z.B. mit rot markiert.

Mögliche Anwendungsbereiche sind einerseits die Forschung und Entwicklung, bei der sie zur Auffindung von optimierbaren Stellen z.B. eines Flugzeugs diehnen können.
Andererseits sind natürlich auch militärische Einsatzzwecke denkbar.


\section{Schallortung}
Die Schallortung mit Hilfe von zwei Mikrophonen ist komplizierter, als mit einem ganzen Array, auch wenn dadurch weniger Rechenarbeit notwendig ist.
Man benötigt zwei Mikrophone, welche man an ein Oszilloskop anschließen muss.
Zunächst wird die Wellenlänge der reinen Sinus-Schallquelle bestimmt, in dem man die zwei Mikrophone dicht nebeneinander hält, so dass zunächst kein Phasenunterschied feststellbar ist.
Dann wird das eine relativ zu dem anderen bewegt, wodurch ein Phasenunterschied auftritt.
Nach einer Wellenlänge ist dieser (Modulo $2\pi$) wieder verschwunden.
Dann kann man den Abstand der Mikrophone bestimmen und damit die Wellenlänge.
Heraus kam eine Wellenlänge von $\lambda=1.7\si{\centi\meter}$, was bei einer Schallgeschwindigkeit von $c=340\si{\meter\per\second}$ einer Frequenz $\nu=\frac{c}{\lambda}=20\si{\kilo\hertz}$ entspricht.
Für Menschen ist dies unhörbar.

Zur Ortung der Schallquelle werden die zwei Mikrophone parallel auf einer Schiene befestigt.
Möchte man eine mögliche Schallquelle überprüfen, so stellt man die Schiene senkrecht zur Achse der Quelle.
Da der Abstand zwischen Quelle und Mikrophonen groß im Vergleich zur Ausdehnung der Quelle ist, können die Wellen als eben angenommen werden.
Dies bedeutet, dass die Wellenfronten in guter Näherung als parallele Geraden ankommen.
Kommt der Schall tatsächlich von dieser Quelle, so schiebt man das Mikrofon entlang der Schiene nur auf der selben Wellenfront hin und her.
Es gibt keinen Phasenverschub.
Kommt der Schall jedoch von einer anderen Quelle, so wird das eine Mikrophon eine andere Phase aufweisen.

\subsection{Schallordnung und Winkelauflösung beim Menschen}
Der Mensch ortet Schall auf eine ähnliche weise.
Kommt ein Schallsignal von der Seite an, so empfängt das dichtere Ohr den Ton einen Sekundenbruchteil früher, als das andere.
Dies kann das Gehirn interpretieren als Richtung, obwohl die "`Signalverarbeitung"' im Gehirn deutlich länger dauert, als die Laufzeitunterschiede.

Möchte man feststellen, wie groß die Winkelauflösung einer Person ist, so kann man zwei (möglichst) gleiche Schallquellen $A$ und $B$ im Abstand $s$ positionieren.
Die Versuchsperson steht im Abstand $l$ von den beiden entfernt.
Nun geben die Signalquellen abgesprochene Muster vor (z.B. $ABA$ oder $BAA$) und die Aufgabe der Testperson ist es, zu bestimmen welche Reihenfolge herrschte.

Bei uns war die Länge $l=8.5\si\meter$ und bei unseren zwei Testpersonen lag $s_\text{min}$ zwischen $3.4^\circ$ und $\leq2^\circ$ (bei dieser Messung hatten die Schallquellen keinen Platz mehr, um dichter zueinander zu kommen).
Falls bessere Bedingungen herrschen (weniger störende Objekte im Raum oder konzentriertere Testpersonen), lassen sich normalerweise noch kleinere Winkel auflösen.

Interessanterweise hat man durch Studien herausgefunden, dass der Mensch gerade in dem Frequenzbereich, in dem er sich am meisten Unterhält ($\nu\approx2\si{\kilo\hertz}$) das schlechteste Auflösungsvermögen besitzt.
Dies scheint evulutionär erstmal fragwürdig, ist aber physikalisch sinnvoll, da es einer Wellenlänge von $17\si{\centi\meter}$ entspricht, was dem Abstand der Ohren entspricht.


\section{dB-Addition}
Die Lautstärke einer Schallquelle in dB wird durch die verursachten Intensitätsschwankungen des Drucks ($p^2$) berechnet:
\begin{align*}
	L=10\log_{10}\left( \frac{\Delta p^2}{p_0^2} \right)\si{\deci\bel}, \qquad p_0=2\cdot10^{-5}\si\pascal\, .
\end{align*}
Dabei ist $p_0$ ein Referenzdruckunterschied, den man vermutete als Hörschwelle bei $1\si{\kilo\hertz}$.
Verdoppelt man die Anzahl der Schallquellen, so steigt der Schallpegel um $10\cdot\log2\;\si{\deci\bel}\approx3\si{\deci\bel}$, da man die doppelte Schallintensität hat ($2\cdot\Delta p^2$).
Unsere Messungen zeigten dies ungefähr.
Dass es nicht das genaue Ergebnis war, liegt vermutlich daran, dass dieses Gesetz nur für Punktquellen gilt, wir aber relativ dicht an den Lautsprechern waren.
Außerdem lag unser Schallmessgerät auf dem Fußboden, der wiederum Schall reflektiert, was ebenfalls zu anderen Ergebnissen führt.
Wenn man den Abstand zwischen Sender und Empfänger verdoppelt, so veringert sich die Lautstärke um $\SI6{\deci\bel}$, da $\Delta p\propto r^{-2}\;\Rightarrow\;\Delta p^2\propto r^{-4}$ ist.
Auch dies war in guter Näherung zu erkennen.

Interessant ist jedoch der Fall, bei dem man zwei Schallquellen nimmt und sie gegenphasig polt.
Rein theoretisch sollte durch destruktive Interferenz die Lautstärke verschwinden.
Dieses Prinzip wird zum Beispiel bei Antischall-Kopfhörern verwendet.
Allerdings zeigte sich in unserem Versuch, dass dies nicht perfekt gelang.
Jedoch ließ sich das Geräusch geringfügig minimieren.
Dies ist auch in den erwähnten Kopfhörern der Fall, da es natürlich noch schwerer ist, wenn die beiden Schallquellen nicht vom selben Rechner aus gesteuert werden.
Da in der Realität der zu eliminierende Ton kein Sinus-Ton ist, sondern relativ zufällig und schnell ändernd, und für die kleinen Wellenlängen die Phasenanpassung auch extrem präzise sein muss, dämpfen die Kopfhörer meist nur die niederfrequenten Geräusche.
Die hohen Töne lassen sich jedoch besser durch das Material des Kopfhörers selber abschirmen, da ihre Dämpfung wesentlich stärker ist.
Dies konnte man zum Beispiel auch Silvester sehr gut hören, da die Knaller, die weit voneinem gezündet wurden (am besten hörbar war es aus ca. $5\si{\kilo\meter}$ Entfernung) sehr viel dumpfer klangen, als dichter explodierende.





\bibliography{literatur}
\bibliographystyle{babalpha}
\end{document}
