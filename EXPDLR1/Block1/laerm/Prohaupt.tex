% Für Bindekorrektur als optionales Argument "BCORfaktormitmaßeinheit", dann
% sieht auch Option "twoside" vernünftig aus
% Näheres zu "scrartcl" bzw. "scrreprt" und "scrbook" siehe KOMA-Skript Doku
\documentclass[12pt,a4paper,titlepage,headinclude]{scrartcl}

%%%%%%%%%%%%%%%%%%%%%%%%%%%%%% Formatierung %%%%%%%%%%%%%%%%%%%%%%%%%%%

%keine Einrückung nach leerzeile
\parindent0pt

% Für Kopf und Fußzeilen, siehe auch KOMA-Skript Doku
\usepackage[komastyle]{scrpage2}
\pagestyle{scrheadings}
\setheadsepline{0.5pt}[\color{black}]
\automark[section]{chapter}

%Zitate und Literaturverzeichnis
\usepackage[backend=bibtex,natbib=true,sorting=nyt,style=numeric-comp]{biblatex}
\usepackage[babel,german=quotes]{csquotes}
\bibliography{literatur}

%Zur vernünftigen Dekodierung
\usepackage[T1]{fontenc} %
\usepackage[utf8]{inputenc} %utfx8
\usepackage[ngerman]{babel} %

%Interaktives Dokument
\usepackage[pdfpagelabels=true]{hyperref}%

%Für wissenschaftliches Zitieren
%\usepackage{natbib}

%Schriftarten
%\usepackage{lmodern} %

%Formatierung für Kof- und Fußzeile. Hier gilt entweder ... oder ...!!

%Für eigenen Zeilenabstand
\usepackage{setspace} %

%Für die Seitenformatierung
\usepackage{lscape} %
\usepackage{multicol} %
\usepackage{wallpaper} %

%Styling Inhaltsverzeichnis
\usepackage{tocloft} %

% Zur Formatierung für Kopf und Fußzeilen. Im Allgemeinen ist scrpage2 besser als fancyhdr
\usepackage{scrpage2}
\pagestyle{scrheadings}
\setheadsepline{0.5pt}[\color{black}]

%Einstellungen für Figuren- und Tabellenbeschriftungen
\setkomafont{captionlabel}{\sffamily\bfseries}
\setcapindent{0em} 


%%%%%%%%%%%%%%%%%%%%%%%%%%%%%% Mathematisches %%%%%%%%%%%%%%%%%%%%%%%%%%%

%Pakete für Mathesymbole
\usepackage{latexsym,exscale,stmaryrd} %
\usepackage{amssymb, amsfonts, amstext} %
\usepackage{amsmath, mathtools, amsthm} %

%align nummerierung
\numberwithin{equation}{subsection}

% Weitere Symbole
\usepackage[nointegrals]{wasysym} %
\usepackage{eurosym} %
\usepackage{textcomp} %

%\usepackage{ucs} %

%Für vernünftige Einheiten 
\usepackage[separate-uncertainty, exponent-product = \cdot]{siunitx}
%\usepackage[thinspace,thinqspace,amssymb]{SIunits} %
\usepackage{icomma} %
\usepackage{nicefrac}%

%SI-Einheiten
\usepackage{siunitx}

%%%%%%%%%%%%%%%%%%%%%%%%%%%%%% Grafiken & Tabellen %%%%%%%%%%%%%%%%%%%%%%%%%%%
% Text umfließt Graphiken und Tabellen
% Beispiel:
% \begin{wrapfigure}[Zeilenanzahl]{"l" oder "r"}{breite}
%   \centering
%   \includegraphics[width=...]{grafik}
%   \caption{Beschriftung} 
%   \label{fig:grafik}
% \end{wrapfigure}
% Mehrere Abbildungen nebeneinander
% Beispiel:
% \begin{figure}[htb]
%   \centering
%   \subfigure[Beschriftung 1\label{fig:label1}]
%   {\includegraphics[width=0.49\textwidth]{grafik1}}
%   \hfill
%   \subfigure[Beschriftung 2\label{fig:label2}]
%   {\includegraphics[width=0.49\textwidth]{grafik2}}
%   \caption{Beschriftung allgemein}
%   \label{fig:label-gesamt}
% \end{figure}

%Subfigure nur mit PDF statt Bildern einfügen:
\usepackage{adjustbox}
%\begin{figure}[h]
%  \centering
%  \subfigure[Caption1\label{fig:bild1}]
%  {\begin{adjustbox}{width=0.44\linewidth}\input{bild1}\end{adjustbox}}
%  \hfill
%  \subfigure[Caption2\label{bild2}]
%  {\begin{adjustbox}{width=0.44\linewidth}\input{bild2}\end{adjustbox}}
%  \hfill
%  \subfigure[Caption3\label{bild3}]
%  {\begin{adjustbox}{width=0.44\linewidth}\input{bild3}\end{adjustbox}}
%  \caption{Gesamtcaption}
%  \label{fig:gesamtlabel}
%\end{figure}

%\usepackage{subfigure}




% Caption neben Abbildung
% Beispiel:
% \sidecaptionvpos{figure}{"c" oder "t" oder "b"}
% \begin{SCfigure}[rel. Breite (normalerweise = 1)][hbt]
%   \centering
%   \includegraphics[width=0.5\textwidth]{grafik.png}
%   \caption{Beschreibung}
%   \label{fig:}
% \end{SCfigure}

%Einstellungen für Figuren- und Tabellenbeschriftungen
\setkomafont{captionlabel}{\sffamily\bfseries}
\setcapindent{0em}

%Fuer mehr Platz in den Tabellen
\usepackage{cellspace} %mehr Platz in Tabellen
\addtolength\cellspacetoplimit{3pt}
\newcommand\myhline[1][2pt]{\\[#1]\hline}

%Zum Einbinden von GRafiken
\usepackage{graphicx}% [pdflatex]
\usepackage{xcolor}%

%Für textumflossene Grafiken
\usepackage{wrapfig} %

%Für subfigure
\usepackage{caption}
\usepackage{subcaption}

% Caption neben Abbildung
\usepackage{sidecap}

%Für URLs
\usepackage{url}%

%Zum Einbinden von Quelltext
\usepackage{listings-ext} %

%Für chemische Formeln
\usepackage{chemfig} %
%Für chemische Formeln (von www.dante.de)
%% Anpassung an LaTeX(2e) von Bernd Raichle
\makeatletter
\DeclareRobustCommand{\chemical}[1]{%
  {\(\m@th
   \edef\resetfontdimens{\noexpand\)%
       \fontdimen16\textfont2=\the\fontdimen16\textfont2
       \fontdimen17\textfont2=\the\fontdimen17\textfont2\relax}%
   \fontdimen16\textfont2=2.7pt \fontdimen17\textfont2=2.7pt
   \mathrm{#1}%
   \resetfontdimens}}
\makeatother
%erzwinge Fussnote auf selber Seite
\interfootnotelinepenalty=1000

%Zusätzliche Boxen
\usepackage{fancybox}

%\usepackage{framed}
%\usepackage{mathmode}
%\usepackage{empheq}

%Für variable Referenzen
\usepackage{varioref}

%Für Tabellen mit fester Gesamtbreite und variabler Spaltenbreite (im Gegensatz zu tabular)
\usepackage{tabularx}
%\newcommand{\ltab}{\raggedright\arraybackslash} % Tabellenabschnitt linksbündig
%\newcommand{\ctab}{\centering\arraybackslash} % Tabellenabschnitt zentriert
%\newcommand{\rtab}{\raggedleft\arraybackslash} % Tabellenabschnitt rechtsbündig


%Für Gleitobjekte
\usepackage{float} %Für H-Option

\usepackage{multirow} % Zellen von Tabellen zusammenfassen
\usepackage{booktabs} % verschoenert Tabellen
\usepackage{fixltx2e} % Repariert einige Dinge in Bezug auf das setzen von Gleitobjekten http://ctan.org/pkg/fixltx2e
\usepackage{stfloats} % Bei Gleitobjekten (figure,table,...) die ueber zwei Spalten gesetzt werden (Umgebung figure*), funktioniert [tb] http://ctan.org/pkg/stfloats
\usepackage{rotating} % Wird für Text und Grafiken benötigt, die um einen Winkel gedreht werden sollen



%%%%%%%%%%%%%%%%%%%%%%%%%%%%%% Kommandodefinitionen %%%%%%%%%%%%%%%%%%%%%%%%%%%

%Zur Korrektur und Kommentierung
\newcommand{\comment}[1]{\marginpar{\tiny{\textcolor{red}{#1}}}} % ermoeglicht kleine Kommentare am Seitenrand: \comment{Fehler?}
\newcommand{\Comment}[1]{\textcolor{red}{#1}}

%Zur Formatierung in der Matheumgebung
\renewcommand{\t}{\ensuremath{\rm\tiny}} % Tiefgestellter Text in der Matheumgebung wird schoener mit: $\Phi_{\t{Text}}$
\renewcommand{\d}{\ensuremath{\mathrm{d}}} % Die totale Ableitung ist stets aufrecht zu setzen: \d
\newcommand{\diff}[3][]{\ensuremath{\frac{\d^{#1}#2}{\d#3^{#1}}}} % einfache Ableitung nach x: $\ddx{\Phi}$
\newcommand{\pdiff}[3][]{\ensuremath{\frac{\partial^{#1}#2}{\partial#3^{#1}}}} % wie gesprochen, eine partielle Ableitung: \del
\newcommand{\aeqiv}{\ensuremath{\qquad \Longleftrightarrow \qquad}} % Eine Aequivalenz
\newcommand{\folgt}{\ensuremath{\qquad \Longrightarrow \qquad}} % Ein Folgepfeil mit Abstaenden
\newcommand{\corresponds}{\ensuremath{\mathrel{\widehat{=}}}} % Befehl für "Entspricht"-Zeichen
\newcommand{\mi}[1]{\ensuremath{\mathit{#1}}} % italics für griechische Buchstaben in Matheumgebung

%Um nicht so viel schreiben zu müssen...
\newcommand{\bs}[1]{\boldsymbol{#1}}
\newcommand{\ol}[1]{\overline{#1}}
\newcommand{\wtilde}[1]{\widetilde{#1}}
\newcommand{\mrm}[1]{\mathrm{#1}}
\newcommand{\mbf}[1]{\mathbf{#1}}
\newcommand{\mbb}[1]{\mathbb{#1}}
\newcommand{\mcal}[1]{\mathcal{#1}}
\newcommand{\mfrak}[1]{\mathfrak{#1}}

%Abkürzungen
\newcommand{\zB}{z.\,B.\ }
\newcommand{\bzw}{b.\,z.\, w.\ }
\newcommand{\Dh}{d.\,h.\ }
\newcommand{\Gl}{Gl.\ }
\newcommand{\Abb}{Abb.\ }
\newcommand{\Tab}{Tab.\ }

%Farbige Box um eine Formel
%Anwendung:
%\eqbox{
%  \begin{equation}
%    ...
%  \end{equation}
%}
\newcommand{\eqbox}[1]{
  \colorbox{gray!30}{\parbox{\linewidth}{#1}} 
}

%Im Text
\newcommand{\engl}[1]{engl. \textit{#1}}
\newcommand{\zitat}[1]{\footnote{#1}}
\newcommand{\person}[1]{\textsc{#1}}


%Matheoperatoren
\DeclareMathOperator{\tr}{tr}
\DeclareMathOperator{\sgn}{sgn}
\DeclareMathOperator{\diag}{diag}
\DeclareMathOperator{\const}{const}
\DeclareMathOperator{\grad}{grad}
\DeclareMathOperator{\rot}{rot}
\DeclareMathOperator{\divz}{div}


%%%%%%%%%%%%%%%%%%%%%%%%% Quellcode - Formatierung %%%%%%%%%%%%%%%%%%%%%%%%%%%%%%%%%%%%%%

%Um auch Umlaute in den Kommentaren auswerten zu können
\lstset{
literate = {Ö}{{\"O}}1 {Ä}{{\"A}}1 {Ü}{{\"U}}1 {ß}{{\ss}}2 {ü}{{\"u}}1
           {ä}{{\"a}}1 {ö}{{\"o}}1
}

%Formatierung des Quellcode
\lstset{
language=C++,
basicstyle=\footnotesize\ttfamily,
keywordstyle=\bfseries\color{blue},
stringstyle=\color{red},
commentstyle=\itshape\color{green!60!black},
emphstyle = \bfseries\color{red!80!green!60!blue}
%identifierstyle=,
}

%Zum Hervorheben bestimmter Begriffe (z.B. eigene Klassen, etc.)
%\lstset{
%emph = {vector, iterator, std, ostream, istream , ofstream, ifstream, fstream, cmath}
%}

%Nummerirung
\lstset{
numbers=left,
numberstyle=\tiny,
stepnumber=2,
numbersep=5pt,
frame=single,
breaklines=true
framesep=5pt,
numbersep=8pt,
breakindent=3ex
}

%Einbunden über
%\lstinputlisting[caption={blablabla}, language=C++]{name.cpp}

\begin{document}
%Autor, etc.
\newcommand{\titel}{Lärm}
\newcommand{\praktikant}{Kevin Lüdemann}
\newcommand{\email}{
      \href{mailto:kevin.luedemann@stud.uni-goettingen.de}
           {kevin.luedemann@stud.uni-goettingen.de} }
\newcommand{\durchfuehrungsdatum}{07.12.2015}
\newcommand{\abgabedatum}{02.01.2016}

%Metainformationen
\hypersetup{
      pdfauthor = {\praktikant~ },
      pdftitle  = {\titel},
      pdfsubject = {\titel}
}

\begin{titlepage}
\centering
\textsc{\Large Experiementelle Verfahren der Strömungsmechanik,\\[1.5ex] DLR Göttingen}

\vspace*{2.5cm}

\rule{\textwidth}{1pt}\\[0.5cm]
{\huge \bfseries
  \titel}\\[0.5cm]
\rule{\textwidth}{1pt}

\vspace*{2.5cm}

\begin{Large}
\begin{tabular}{ll}
Praktikanten: &  \praktikant,\\
Email:	& \email\\
Durchgeführt am: & \durchfuehrungsdatum\\
Abgegeben am: & \abgabedatum\\
\end{tabular}
\end{Large}
\vspace*{0.8cm}
\end{titlepage}


\pagenumbering{arabic}

\newpage

\section{Schalortung}
\subsection{Theorie}
\label{sec:theo}
Schall breitet sich als Wellen aus.
Diese besitzen eine Wellenlänge $\lambda$ und eine Frequenz $\omega$.
Die Wellenlänge ist der Abstand von einem Intensitätsmaximum zum nächsten.
Desweiteren breitet sich eine Welle in Luft mit der Schallgeschwindigkeit aus.
Diese beträt in der Luft $c=340\si{\meter\per\second}$.
Somit ergibt sich eine Beziehung zwischen der Wellenlänge und der Frequenz
\begin{align}
	\lambda=\frac{c}{\omega}\label{eq:lw}
\end{align}
über die Ausbreitungsgeschwindigkeit.\\
Das Menschliche Ohr kann Frequenzen in einem Bereich von minimal $20\si{\hertz}$ bis zu maximal $20\si{\kilo\hertz}$ aufnehmen.
Dieser Bereich schränkt sich im Alter allerdings auch ein.\\
Im Raum breiten sich Wellen als Wellenfronten aus.
Diese bilden ungestört ebene Wellenfronten, sprich parallel verlaufende Maxima im Raum.
Dies ist nur von dem Weg, den die Welle zurückgelegt hat und von den darin beinhalteten Störungen abhängig.
Bei einer kleinen Wellenlänge im $\si{\centi\meter}$ Bereich, reichen schon einige $\si{\deci\meter}$ Weg um einen Ebene Wellenfront zu erhalten.

\subsection{Experiment}
Als experimenteller Aufbau wird ein Lautsprecher an einen Frequenzgenerator angeschlossen und auf eine nicht hörbare Frequenz gestellt.
Stehen jetzt mehrer Lautsprecher nebeneinander, kann ein Mensch nicht sagen, welcher von diesen den Ton erzeugt.
Als Hilfsmittel, werden 2 Mikrofone verwendet, die die gleiche Charakteristik in bezug auf Schallaufnahme haben.
Die von dem Mikrofon aufgenommenen Schwingungen, werden dann auf z.B. einem Oszilloskop sichtbar gemacht.
Mithilfe des Oszilloskops ist es möglich sich den Phasenverschub der beiden Wellen anzuschauen.
Mit dieser Information ist es möglich die Schallquelle zu bestimmen.\\
Um sie genau zu identifizieren, muss allerdings noch die Ausrichtung der Mikrofone zueinander beachtet werden.
Hierbei ist wichtig, dass beide Mikrofone die gleiche Wellenfront einfangen.
Bei hohen Frequenzen ist die Wellenlänge sehr klein und liegt im $\si{\centi\meter}$ Bereich.
Um dies zu bewerkstelligen müssen die Mikrofone auf einer Schiene befestigt sein und können in zwei Richtungen zueinander verschoben werden.
Einmal kann der Abstand zwischen den beiden verändert und zum anderen können die Mikrofone auch in der Tiefe verschoben werden.
Aufgrund der für die Wellenlänge großen Entfernung, kann angenommen werden, dass sich die Wellen als Ebene Welle ausbreiten.
Sind beide Mikrofone auf eine Schallquele ausgerichtet und die beiden Wellen auf dem Oszilloskop im Phase, wird der Abstand der beiden Mikrofone verändert.
Verändert sich der Phasenunterschied nicht wärend des Verschiebens, so sind die Mikrofone auf dem aktiven Lautsprecher ausgerichtet und bewegen sich auf der selben Wellenfront.
Ist dies nicht der Fall, sieht man wie sich die Phase immer weiter verschiebt, bis sie im Laufe der Messung die Fronten ein paar mal wieder in Phasen befinden.\\
Um die Frequenz zu bestimmen nutzt man das gleiche Verfahren in der Tiefe.
Hierzu werden die Mikrofone so dicht, wie nur möglich zusammen gebracht und durch ausrichten gleichsinnig gemacht.
Verschiebt man jetzt eines in der Tiefe gegen das andere, so sollte sich die Phase ändern, bis beide Wellen wieder in der gleichen Phase sind.
Ist dies erreicht, kann gemessen werden in welchen Abstand sich die beiden zueinander befinden, dies ist dann die Wellenlänge.
Bei uns waren es etwa $\lambda=1.7\si{\centi\meter}$, was nach Gleichung \ref{eq:lw} einer Frequenz von $\omega=20\si{\kilo\hertz}$ entspricht.\\
Eine andere Möglichkeit die Frequenz zu bestimmen nutzt das Oszilloskop.
Hierbei wird die Frequenz einfach abgelesen anhand der Scala auf dem Bildschirm und der Frequenzeinstellung selbt.
Dies ergab bei und ebenfalls eine Frequenz von $\omega=20\si{\kilo\hertz}$.


\section{Minimaler Hörabstand}
Um den mit den Ohren minimal noch Hörbaren Abstand zu messen, wird wieder vorausgesetzt, das sich der Schall, wie im Kapitel \ref{sec:theo} beschrieben verhällt.
In diesem Experiment werden aber Hörbare Schalquellen verwendet, die ein ununterscheidbares Muster aussenden.
Die Schallquellen sind 2 Experimentatoren und der Empfänger ist ein anderer Experimentator.
Die beiden Schallquellen, nennen wir sie A und B, sollen möglichst auf die gleiche weise Schall erzeugen, damit nicht zu erkenne ist, wer von beiden in welcher Freuquenz oder Dauer Schall erzeugt.\\
Für das Experiment stellen sich die beiden Quellen weit von einander entfernt in einem Rechtwinkligen Dreieck zueinander und zum Emfänger, um später die Winkelauflösung zu berechnen.
Anschließend wird von den Quellen in zufälligen Mustern Schall erzeugt und der Empfänger muss die Muster Korrekt wiedergeben.
Hat der Empfänger zwei mal die Muster richtig erkannt wandert der Äußere der Beiden um einen Schritt weiter zum anderen hin.
Hat der Empfänger einmal das falsche Muster genannt, so rückt der Äußere wieder einen halben Schritt vom anderen Weck.
Das rechtwinklige Dreieck muss in jedem Schritt aufrecht erhalten werden.
Mit diesem sogenannten Schießverfahren kann jetzt der minimal Hörbare Abstand ermittelt werden.\\
Bei dem von uns durchgefürtem Verfahren hatten wir einen Abstand von $8\si{\meter}$ zwischen dem Empfänger und den Sendern.
Es ergab sich nach dem Verfahren ein Abstand der beiden Sender von minimal $35\si{\centi\meter}$.
Über den Satz den Pythagoras ergibt sich ein Winkel von $\tan{\varphi}=\frac{\text{Abstand A und B}}{\text{Abstand Empfänger Sender}}$.
Die getestete Winkelauflösung beträgt somit $\varphi=2.5\si{\degree}$.

\newpage
%\nocite{*} %sorgt dafuer, dass alles ausgegeben wird
\printbibliography[heading=bibintoc]
\end{document}
